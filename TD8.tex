\documentclass[11pt,a4paper]{article}
\usepackage[utf8]{inputenc}
\usepackage[T1]{fontenc}
\usepackage{tpl} %tdtp
\usepackage{aeguill}
\usepackage{epsfig,graphicx}
\usepackage{subfigure}
\usepackage{eurosym}
\usepackage{enumitem}
\usepackage{hyperref}
\makeatletter


\renewcommand\thesection{Exercice \arabic{section} : }
\renewcommand{\theenumi}{Q.\arabic{section}.\arabic{enumi}}

\sorte{TD}
\siglemat{Python et Open data}
\formation{L2 MIASHS}
\titre{TD8 : Données du réseau STAR}
\begin{document}


Le réseau de transports en commun de l'agglomération rennaise (la STAR) fournit des données relatives à ses services en libre accès via une API de données.
Dans ce TD, vous effectuerez des requêtes à cette API en utilisant le module \verb+requests+.

\section*{Préambule}
\begin{enumerate}
    \item Dans votre dossier PythonOpenData créez un sous-dossier TD7.
    \item Lancer l'éditeur Visual Studio Code.
    \item Dans Visual Studio Code, ouvrir le dossier PythonOpenData/TD7 et créez un fichier td7.py dans lequel vous écrirez votre code. 
\end{enumerate}

Pour ce TD, il est conseillé d'importer le module \verb+pprint+ qui permet d'afficher de manière claire les dictionnaires :

\begin{verbatim}
from pprint import pprint

[...]
pprint(mon_joli_dictionnaire)
\end{verbatim}

\section{Accès aux données}

\begin{itemize}
    \item Se rendre sur le site de la STAR (\url{https://data.explore.star.fr/explore/})
            et trouver l'API indiquant les prochains passages de métro rennais.
    \item Cliquez sur l'onglet "API" pour accéder aux options de requête.
    \item Essayez notamment d'ajouter le \emph{facet} \verb+depart+ et notez le format de date
            utilisé (un \emph{facet} est un attribut dont on demande explicitement qu'il soit 
            présent dans la réponse pour tous les résultats retournés).
    \item En utilisant l'interface d'édition de requêtes de l'API de la STAR, composez une requête qui permette :
    \begin{itemize}
        \item d'afficher les attributs \verb+depart+, \verb+destination+ et \verb+nomarret+ pour les résultats 
            retournés (ajout de ces attributs à la liste des \emph{facets})
        \item de ne conserver que les passages pour lesquels l'attribut \verb+precision+ vaut 
        \verb+Temps réel+ (valeur à spécifier dans la catégorie \emph{refine})
        \item de forcer les dates à être spécifiées dans le fuseau horaire \verb+Europe/Paris+
        \item de retourner les 100 prochains passages.
    \end{itemize}
    \item Notez l'URL générée (clic droit sur le lien du bas de la page, puis "Copier le lien").
\end{itemize}

\section{Travail spécifique sur les dates}

\begin{quote}
    Petit point sur :
    
    \textbf{UTC, Temps Universel Coordonné, (Coordinated Universal Time)}
    est une échelle de temps adoptée comme base universelle.
    En France, nous avons une heure d'avance sur ce temps de référence.
    Pour représenter une date complète utilisable à l'international
    l'API de la Star utilise le modèle suivant
    \verb|"jourThoraire+timezone"| en trois parties avec pour séparateurs \verb+"T"+ et \verb|"+"|
    où :
    
    \begin{itemize}
        \item jour = \verb+"aaaa-mm-jj"+ ,
        \item  horaire = \verb+"hh:mn:sc"+,
        \item  timezone \verb+IN {"00:00", "01:00"...}+
    \end{itemize}

    Ainsi :
    
    \begin{itemize}
        \item si timezone vaut \verb+"01:00"+ il s'agit de l'heure "en France",
        \item si timezone vaut \verb+"00:00"+, il y a une heure de retard
    \end{itemize}
\end{quote}

\begin{itemize}
    \item Écrire une fonction qui prend en entrée une chaîne de caractères 
        représentant une date dans le format vu à la requête précédente (exemples : 
        \verb+"2021-11-25T09:01:52+01:00"+  ou \verb+"2021-11-25T09:01:52+00:00"+)
        et retourne une date de sortie ayant les caractéristiques suivantes:
        \begin{itemize}
            \item du type date de Python (ce qui permet la réalisation de calculs, de comparaison...) ,
            \item privé de timezone et
            \item son horaire doit être celui du fuseau horaire français.
        \end{itemize}
    \item Écrivez une fonction qui retourne la liste de tous les passages de métro à
        venir. Cette fonction fera une requête API, en limitant le nombre de résultats à 100 lignes. 
        La liste retournée par cette fonction contiendra des dictionnaires composés de 3 clés : 
        \verb+depart+ (contenant l'heure de départ au format \verb+datetime+), \verb+destination+ et \verb+nomarret+ 
        et vous ne conserverez que les passages pour lesquels l'attribut \verb+precision+ vaut \verb+Temps réel+.
        **Attention :** pour certains passages, l'attribut \verb+"depart"+ n'existe pas :
        ces passages doivent donc être ignorés.
    \item Écrivez une fonction qui prend en entrée une liste de passages tels que ceux retournés par la question 
        précédente et un délai \verb+t+ en minutes et qui retourne la liste des passages qui auront lieu dans un délai de \verb+t+ 
        minutes après l'instant présent. 
    \item Tester cette fonction en affichant la liste des prochains passages de métro dans 
        les 10 minutes à venir.
\end{itemize}

\section{Pour aller plus loin}

En utilisant le service Open Data de Rennes Métropole (\url{https://data.rennesmetropole.fr/}), écrivez une
fonction qui affiche le nombre total de passages de vélos (même si le nom du jeu de données 
sous-entend qu'il fournit des infos sur les passages de vélos et de piétons, seuls les vélos 
sont comptés) devant chacun des 
compteurs installés dans Rennes (attribut \verb+name+), pour le mois de novembre 2021.

Notez que dans l'interface utilisée, pour filtrer une date par mois 
(c'est-à-dire ne conserver que les enregistrements pour le mois \verb+MM+ de l'année 
\verb+YYYY+), on peut demander que l'attribut \verb+date+ soit de la forme : \verb+YYYY/MM+,
en donnant les valeurs voulues à \verb+YYYY+ et \verb+MM+.

Votre fonction devra afficher une sortie de la forme :

\begin{verbatim}
Le compteur Eco-Display Place de Bretagne a vu passer 48827 vélos en novembre 2021.
Le compteur Rennes Rue d'Isly V1 a vu passer 20703 vélos en novembre 2021.
\end{verbatim}


\end{document}

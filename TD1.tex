\documentclass[11pt,a4paper]{article}
\usepackage[utf8]{inputenc}
\usepackage[T1]{fontenc}
\usepackage{tpl} %tdtp
\usepackage{aeguill}
\usepackage{epsfig,graphicx}
\usepackage{subfigure}
\usepackage{eurosym}
\usepackage{enumitem}
\usepackage{hyperref}
\makeatletter


\renewcommand\thesection{Exercice \arabic{section} : }


\sorte{TD}
\siglemat{Python et Open data}
\formation{L1 MIASHS}
\titre{TD1 : API de calcul de trajets}
\begin{document}

\section*{Travail à préparer avant la séance}
\begin{itemize}
    \item Créez-vous un compte GraphHopper en cliquant ici (vous utiliserez l’URL \url{http://www.univ-rennes2.fr} pour le champ “site web”)
\item Créez-vous une clé d’API en vous rendant, connecté avec votre compte, à l’adresse \url{https://graphhopper.com/dashboard/#/api-keys}
\item Sauvegarder la clé d'API créée.
\end{itemize}

\section*{Préambule}
\begin{enumerate}
    \item Créer sur votre disque un dossier PythonOpenData dans lequel vous réaliserez l'ensemble des travaux. 
    \item Télécharger sur Cursus l'archive contenant les données sources du TD1.
    \item Décompresser cette archive sur votre disque. 
    Les fichiers doivent se trouver dans PythonOpenData/TD1.
    \item Lancer l'éditeur Visual Studio Code.
    \item Dans Visual Studio Code, ouvrir le dossier PythonOpenData. 
\end{enumerate}

\section*{Installation du module graphh}

Par défaut, le module graphh n’est pas installé sur votre ordinateur ni sur les machines de l’université. Pour pouvoir l’utiliser, il va donc falloir commencer par l’installer. 

Avec Visual Studio Code : 
\begin{itemize}
    \item Ouvrir une fenêtre de terminal par le menu "Affichage - Terminal".
    \item Taper la commande \verb+pip install graphh+.
\end{itemize}

Note : cette action est à effectuer une fois pour toutes sur votre machine personnelle, si vous l’utilisez en TD, ou à chaque début de TD sur une machine de l’université si vous travaillez sur un poste de l’université.

\section{Préparation de la clé d'API}
\begin{enumerate}
    \item Dans le fichier credentials.json, insérer votre clé d'API préalablement créée.
    \item Dans le fichier td2.py, la fonction fournie \verb+acces_cle_api+ permet de lire la clé API dans le fichier credentials.json. 
    Utiliser cette fonction pour afficher votre clé d'API.
\end{enumerate}
        
\section{Prise en main de GraphHopper}
\begin{enumerate}
    \item Créer un clien GraphHopper en utilisant votre clé d'API.
    \item Obtenir les coordonnées GPS de "Rennes Beaulieu" ainsi que de "Rennes Villejean".
    \item Afficher la distance par la route pour relier les deux campus, de Beaulieu vers Villejean, puis dans le sens inverse.
    \item Afficher la durée pour relier les deux campus, dans un sens puis dans l'autre.
\end{enumerate}

\section{Définition de fonctions}
\begin{enumerate}
    \item Créer une fonction \verb+distance_lieux+ qui prend en entrée un client GraphHopper, et deux chaines de caractères représentant des lieux, et qui renvoie la distance entre ces deux lieux.
    \item Tester cette fonction pour calculer la distance de Nantes à Rennes.
    \item Créer une fonction \verb+duree_lieux+ qui prend en entrée un client GraphHopper, et deux chaines de caractères représentant des lieux, et qui renvoie la duree du trajet entre ces deux lieux.
    \item Tester cette fonction pour calculer la duree du trajet de Nantes à Rennes.
\end{enumerate}


\section{Voyage à étapes}
On définit un voyage comme une liste de lieux (chaines de caractère). Chaque lieu de la liste est une étape du voyage.

Pour toutes les fonctions demandées, penser à chaque fois à utiliser d'autres fonctions existantes !

\begin{enumerate}
    \item Une variable \verb+voyage1+ est définie dans le fichier source. Définir un second voyage de votre choix.
    \item Définir une fonction \verb+distance_etapes+ qui prend en entrée un voyage, un client GraphHopper, et renvoie la liste des distances de chaque étape. Tester sur les deux voyages.
    \item Définir une fonction \verb+distance_totale+ qui prend en entrée un voyageun client GraphHopper, et renvoie la distance totale du trajet en passant par chacune des étapes.
\end{enumerate}

\section{Quel co-voiturage ?}

Dans cette partie, on suppose que l’on cherche à déterminer, parmi plusieurs choix possibles, un trajet optimal avec les contraintes suivantes :
\begin{itemize}
    \item les points de départ et d’arrivée sont fixés
    \item on doit récupérer, sur le trajet, une personne en covoiturage et l’on peut choisir entre plusieurs personnes situées à des positions différentes.
\end{itemize}

La question peut par exemple être posée sous la forme suivante : \og 
    Lors d’un trajet Rennes - Marseille, vaut-il mieux (en termes de temps de trajet) prendre quelqu’un à “Paris 14ème arrondissement”, “Lyon 1er arrondissement” ou “Bordeaux” ? \fg

\begin{enumerate}
    \item Écrire une fonction qui prend en entrée :
    \begin{itemize}
        \item un lieu d’origine (sous la forme d’une chaîne de caractères)
        \item une destination (sous la forme d’une chaîne de caractères)
        \item une liste d’options sur le trajet (sous la forme d’une liste de chaînes de caractères)
        \item un client d’API GraphHopper
    \end{itemize}
    et retourne l’option correspondant au trajet le plus court (en temps). En cas d’égalité, votre fonction retournera l’une des options correspondant au trajet le plus court (en temps).

\end{enumerate}

\end{document}

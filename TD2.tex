\documentclass[11pt,a4paper]{article}
\usepackage[utf8]{inputenc}
\usepackage[T1]{fontenc}
\usepackage{tpl} %tdtp
\usepackage{aeguill}
\usepackage{epsfig,graphicx}
\usepackage{subfigure}
\usepackage{eurosym}
\usepackage{enumitem}
\usepackage{hyperref}
\makeatletter


\renewcommand\thesection{Exercice \arabic{section} : }


\sorte{TD}
\siglemat{Python et Open data}
\formation{L1 MIASHS}
\titre{TD2 : Problèmes de trajets, co-voiturages et copains}
\begin{document}

\section*{Préambule}
\begin{enumerate}
    \item Dans votre dossier PythonOpenData, créer un sous-répertoire TD2 dans lequel vous réaliserez l'ensemble des travaux. 
    \item Télécharger sur Cursus l'archive contenant les données sources du TD2.
    \item Lancer l'éditeur Visual Studio Code.
    \item Dans Visual Studio Code, ouvrir le dossier PythonOpenData/TD2. 
    \item Copier le fichier credentials.json du TD1 dans le dossier PythonOpenData/TD2.
\end{enumerate}



\section{Quel co-voiturage ?}

Dans cette partie, on suppose que l’on cherche à déterminer, parmi plusieurs choix possibles, un trajet optimal avec les contraintes suivantes :
\begin{itemize}
    \item les points de départ et d’arrivée sont fixés
    \item on doit récupérer, sur le trajet, une personne en covoiturage et l’on peut choisir entre plusieurs personnes situées à des positions différentes.
\end{itemize}

La question peut par exemple être posée sous la forme suivante : \og 
    Lors d’un trajet Rennes - Marseille, vaut-il mieux (en termes de temps de trajet) prendre quelqu’un à “Paris 14ème arrondissement”, “Lyon 1er arrondissement” ou “Bordeaux” ? \fg

\begin{enumerate}
    \item Écrire une fonction qui prend en entrée :
    \begin{itemize}
        \item un lieu d’origine (sous la forme d’une chaîne de caractères)
        \item une destination (sous la forme d’une chaîne de caractères)
        \item une liste d’options sur le trajet (sous la forme d’une liste de chaînes de caractères)
        \item un client d’API GraphHopper
    \end{itemize}
    et retourne l’option correspondant au trajet le plus court (en temps). En cas d’égalité, votre fonction retournera l’une des options correspondant au trajet le plus court (en temps).

\end{enumerate}

\end{document}

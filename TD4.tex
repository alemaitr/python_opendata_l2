\documentclass[11pt,a4paper]{article}
\usepackage[utf8]{inputenc}
\usepackage[T1]{fontenc}
\usepackage{tpl} %tdtp
\usepackage{aeguill}
\usepackage{epsfig,graphicx}
\usepackage{subfigure}
\usepackage{eurosym}
\usepackage{enumitem}
\usepackage{hyperref}
\makeatletter


\renewcommand\thesection{Exercice \arabic{section} : }


\sorte{TD}
\siglemat{Python et Open data}
\formation{L2 MIASHS}
\titre{TD4 : Restaurants new-yorkais}
\begin{document}

Ce sujet traite des fichiers JavaScript Object Notation (JSON).
Ce format permet de stocker des données structurées, par exemple avec une organisation hiérarchique.

\section*{Préambule}
\begin{enumerate}
    \item Dans votre dossier PythonOpenData, créez un sous-répertoire TD4 dans lequel vous réaliserez l'ensemble des travaux.
    \item Lancer l'éditeur Visual Studio Code depuis Anaconda Navigator.
    \item Dans Visual Studio Code, ouvrir le dossier PythonOpenData/TD4. 
\end{enumerate}

Dans ce TD, on se tourne vers un jeu de données recensant des notes attribuées à des restaurants de la ville de New York.

\section{Chargement des données}

\begin{enumerate}
    \item Sur Cursus, télécharger le jeu de données \verb+NYfood.json+ et l'enregistrer dans le dossier PythonOpenData/TD4.
    \item Visualiser avec Visual Studio Code le contenu du fichier JSON.
    \item Quels sont les attributs (clés de dictionnaires) de ce jeu de données ? Que contient l'attribut \verb+grades+ (qui signifie \verb+notes+ en français) ?
\end{enumerate}

\section{Extraction d'informations élémentaires}

Dans la suite de ce sujet, on nomme \og restaurant\fg ~chaque dictionnaire issu du fichier JSON.
Les questions suivantes nécessitent de manipuler les données depuis Python et le code produit devra être inclus dans un nouveau fichier \verb+td4.py+.

\begin{enumerate}
\item Charger le contenu du fichier \verb+NYfood.json+ dans une liste de dictionnaires, chaque dictionnaire représentant un restaurant.

\item Combien y a-t-il de restaurants dans ce fichier ?

\item Combien y a-t-il de restaurants situés à Manhattan listés dans ce fichier ?

\item Écrire une fonction qui prend en entrée la liste des restaurants et retourne une liste sans doublon des quartiers dans lesquels se situent les restaurants listés dans ce fichier.

\item Écrire une fonction qui prend en entrée la liste des restaurants et affiche, pour chaque restaurant du quartier ``Manhattan'', un récapitulatif de la forme \verb+nom_du_restaurant: adresse+ où l'adresse est composée de \verb+building, street, zipcode New York, USA+

\item Quel est le nombre total de notes attribuées aux restaurants du fichier ?

\item Quelles sont les valeurs possibles de notes attribuées aux restaurants par les évaluateurs ?
\end{enumerate}

\section{Travail spécifique sur les dates}

\begin{enumerate}
    \item Remarquer que les dates associées aux notes n'ont pas été reconnues comme telles
    \item Écrire une fonction qui prend en entrée la liste des restaurants et retourne une version transformée dans laquelle les dates sont codées au format \verb+datetime.datetime+.
    \item Calculer, pour chaque mois de l'année 2014, le nombre total de notes attribuées (on rappelle que les variables de type \verb+datetime+ ont des attributs \verb+year+ et \verb+month+ accessibles via \verb+d.year+ et \verb+d.month+).
\end{enumerate}

\section{Export au format CSV}

Dans cette partie, il va s'agir d'enregistrer un export du contenu du fichier \verb+NYfood.json+ au format CSV, en s'assurant donc que l'on n'ait plus d'attributs imbriqués.

\begin{enumerate}
    \item Écrire une fonction qui prend en entrée le jeu de données précédent et retourne une nouvelle version de ce jeu de données dans lequel l'attribut \verb+grades+ a été supprimé et remplacé par un attribut \verb+n_grades+ qui indique le nombre de notes reçues.
    
    \item Écrire une fonction qui prend en entrée le jeu de données précédent et retourne une nouvelle version de ce jeu de données dans lequel l'attribut \verb+address+ est supprimé et remplacé par deux nouveaux attributs \verb+latitude+ et \verb+longitude+ (l'attribut \verb+coordinates+ est une liste contenant la longitude et la latitude, stockées dans cet ordre).
    
    \item En utilisant les deux fonctions précédentes, exporter cette nouvelle version du jeu de données dans un fichier nommé \verb+NYfood.csv+.
\end{enumerate}

\end{document}

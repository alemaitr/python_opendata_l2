\documentclass[11pt,a4paper]{article}
\usepackage[utf8]{inputenc}
\usepackage[T1]{fontenc}
\usepackage{tpl} %tdtp
\usepackage{aeguill}
\usepackage{epsfig,graphicx}
\usepackage{subfigure}
\usepackage{eurosym}
\usepackage{enumitem}
\usepackage{hyperref}
\makeatletter


\renewcommand\thesection{Exercice \arabic{section} : }
\renewcommand{\theenumi}{Q.\arabic{section}.\arabic{enumi}}

\sorte{TD}
\siglemat{Python et Open data}
\formation{L2 MIASHS}
\titre{TD9 : API Twitter}
\begin{document}

Dans ce TD, vous utiliserez le module `tweepy` pour manipuler des données issues de l'API Twitter.


\section*{Travail à préparer chez vous avant la séance}

Suivez les indications fournies sur le document TD9-péparation, pour créer votre compte Twitter et obtenir les clé API. 

\section*{Préambule}
\begin{itemize}
    \item Créer sur votre disque un dossier TD9 dans PythonOpenData. 
    \item Copier le fichier \verb+credentials.json+ du TD1 dans TD9.
    \item Télécharger sur Cursus l'archive contenant les données sources du TD9, et la décompresser.
    \item Lancer l'éditeur Visual Studio Code.
    \item Dans Visual Studio Code, ouvrir le dossier PythonOpenData/TD9. 
\end{itemize}

\subsection*{Installation du module tweepy}

Sur les ordinateurs de l'Université de Rennes 2, le module `tweepy` n'est pas installé par défaut, vous devez donc l'installer. Pour cela, vous ouvrir le terminal dans Visual Studio Code, puis entrer la ligne : 

\verb+pip install --user tweepy+.

\section{Identification}

Vous allez devoir vous authentifier sur l'API Twitter.Vous ne devrez jamais laisser apparaître vos identifiants dans votre code Python, mais les stocker dans un fichier externe.

\begin{enumerate}
    \item Compléter le fichier \verb+credentials.json+ avec vos identifiants de twitter, pour qu'il ait le format suivant : 
\begin{verbatim}
{
    "twitter": {
        "CONSUMER_KEY": "...",
        "CONSUMER_SECRET": "...",
        "ACCESS_TOKEN": "...",
        "ACCESS_TOKEN_SECRET": "..."
    },
    "GraphHopper": {}
}

\end{verbatim}
où les `"..."` seront remplacés par vos identifiants fournis par l'interface Twitter.

\item Écrire une fonction \verb+client_twitter+ qui prend en entrée le nom d'un fichier json contenant les clés, qui lit les identifiants dans ce fichier et qui retourne un client d'accès à l'API Twitter pour ces identifiants.
\end{enumerate}




\section{Écriture de tweets}
\begin{enumerate}
    \item Écrire une fonction \verb+tweeter+ qui prend en entrée un client Twitter, un message, et qui poste un tweet avec ce message. 
    Cette fonction doit renvoyer l'identifiant du tweet ainsi créé.
    \item Tester cette fonction pour poster un message de votre choix (ex : "Je suis en TP Python"). 
\end{enumerate}

\section{Identifiants utilisateurs}

\begin{enumerate}

    \item Créer un fonction \verb+details_utilisateur+ qui prend en entrée un client et un nom d'utilisateur Twitter, et qui renvoie un dictionnaire contenant le détail des informations de cet utilisateur. 
    
    Par exemple, pour \verb+UnivRennes_2+, on renverra le dictionnaire suivant : 
    \begin{verbatim}
    {'Utilisateur': 'UnivRennes_2', 
    'User_id': 1354601634, 
    'Nom': 'Université Rennes 2', 
    'Lieu': 'Rennes, France', 
    'Description': "Toute l'actualité des #Arts, #Lettres, #Langues,
               #SHS et #STAPS à l'Université #Rennes2", 
    'Date de création':'04/2013'}
    \end{verbatim}
    Notons que la date de création est convertie sous forme d'une chaine de caractère ne contenant que le mois est l'année.

    \item Utiliser cette fonction pour obtenir votre identifiant utilisateur.

\end{enumerate}


\section{Consultation de tweets}

\begin{enumerate}


\end{enumerate}


% \begin{enumerate}
    
    
% \item  Écrire une fonction qui prend en entrée le client d'accès à l'API et retourne la liste des 2 derniers tweets de l'utilisateur identifié.

% \item Écrire une fonction qui prend un tweet en entrée (de type `Status`) et retourne le texte de ce tweet.

% \item En utilisant la fonction de la question précédente, écrire une fonction qui prend une liste de tweets en entrée et retourne la liste des textes des tweets en question.

% \item Écrire une fonction qui prend en entrée un tweet (de type `Status`) et renvoie un tuple contenant retourne l'identifiant et le nom de son auteur.

% \end{enumerate}


\section{Suppression de tweets}

\begin{enumerate}
    \item Écrire une fonction qui prend en entrée la variable d'accès à l'API et une liste d'identifiants de tweets et supprime tous les tweets correspondants.
\end{enumerate}


\end{document}

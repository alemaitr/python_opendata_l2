\documentclass[11pt,a4paper]{article}
\usepackage[utf8]{inputenc}
\usepackage[T1]{fontenc}
\usepackage{tpl} %tdtp
\usepackage{aeguill}
\usepackage{epsfig,graphicx}
\usepackage{subfigure}
\usepackage{eurosym}
\usepackage{enumitem}
\usepackage{hyperref}
\makeatletter


\renewcommand\thesection{Exercice \arabic{section} : }


\sorte{TD}
\siglemat{Python et Open data}
\formation{L2 MIASHS}
\titre{TD9 : API Twitter}
\begin{document}


\section*{Préambule}
\begin{enumerate}
    \item Créer sur votre disque un dossier TD9 dans PythonOpenData. 
    \item Télécharger sur Cursus l'archive contenant les données sources du TD9, et la décompresser.
    \item Lancer l'éditeur Visual Studio Code.
    \item Dans Visual Studio Code, ouvrir le dossier PythonOpenData/TD9. 
\end{enumerate}

\end{document}

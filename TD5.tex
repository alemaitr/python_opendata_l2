\documentclass[11pt,a4paper]{article}
\usepackage[utf8]{inputenc}
\usepackage[T1]{fontenc}
\usepackage{tpl} %tdtp
\usepackage{aeguill}
\usepackage{epsfig,graphicx}
\usepackage{subfigure}
\usepackage{eurosym}
\usepackage{enumitem}
\usepackage{hyperref}
\makeatletter


\renewcommand\thesection{Exercice \arabic{section} : }


\sorte{TD}
\siglemat{Python et Open data}
\formation{L2 MIASHS}
\titre{TD5 : Dénivelés de randonnées}
\begin{document}

Ce sujet est une synthèse des travaux avec GraphHopper et avec Json. 

Le fichier mini-rando\_gps.json fournit des séries de positions GPS correspondant à des traces GPS de sorties randonnée de M. Toulemonde. On cherchera dans ce TD à écrire un programme calculant les dénivelés cumulés positif et négatif de chacune de ces randonnées. Pour cela, vous utiliserez l’API GraphHopper via le module Python graphh.

\section*{Préambule}
\begin{enumerate}
    \item Dans votre dossier PythonOpenData, créez un sous-répertoire TD5 dans lequel vous réaliserez l'ensemble des travaux.
    \item Lancer l'éditeur Visual Studio Code.
    \item Dans Visual Studio Code, ouvrir le dossier PythonOpenData/TD5. 
    \item Télécharger le fichier mini-rando\_gps.json dans le dossier PythonOpenData/TD5.
  \item Copier le fichier credentials.json du TD1 dans le dossier PythonOpenData/TD5.
  \item Créer un fichier td5.py dans lequel vous écrirez le code du TD. 
\end{enumerate}

\section{Altitudes avec GraphHopper}
\begin{enumerate}
    \item Écrire une fonction qui prend en entrée un lieu et un client GraphHopper, et affiche l'altitude de ce lieu.  Tester en affichant l'altitude de Rennes, de Saint-Malo puis de Chamonix.
    \item Ecrire une fonction qui prend en entrée une liste de positions GPS (chacune codée sous la forme d’un dictionnaire comme précisé plus bas) et un client GraphHopper et retourne une liste d’altitudes. Vous pourrez utiliser l’exemple suivant pour vos tests :
\begin{verbatim}
    lst_gps = [
      {"lng": -1.426533, "lat": 48.005135},
      {"lng": -1.418127, "lat": 47.986058},
      {"lng": -1.427611, "lat": 47.989871},
      {"lng": -1.430202, "lat": 48.000354}
    ]
\end{verbatim}
\end{enumerate}

\section{Dénivelés}

En utilisant la fonction écrite à la question précédente, écrire
une fonction qui prend en entrée une liste de positions GPS et un client GraphHopper et retourne la somme des dénivelés positifs (d’une part) et négatifs (d’autre part). 

Par exemple, si on a une liste de coordonnées GPS pour lesquelles on a obtenu les altitudes suivantes : \verb+[35.29, 64.36, 48.78, 35.81]+ on devrait retourner la paire de valeurs : \verb+(29.07, 28.55)+

\section{Analyse de randonnées}

En utilisant la fonction écrite à la question précédente, écrire une fonction qui prend en entrée un nom de fichier JSON (contenant des informations sur diverses randonnées) et un client GraphHopper et affiche, pour chaque randonnée, son nom (attribut "name") et la somme de ses dénivelés positifs (d’une part) et négatifs (d’autre part). 

Pour le fichier mini-rando\_gps.json, on doit obtenir (après quelque temps, le nombre de requêtes à effectuer étant assez grand) une sortie du type :
\begin{verbatim}
TraceGPS Le long de la quincampoix - Pire-sur-Seiche 
    - Dénivelé positif cumulé : 21.87
    - Dénivelé négatif cumulé : 21.87
TraceGPS Issued  Messac - CIRCUIT DU PORT 
    - Dénivelé positif cumulé : 12.71
    - Dénivelé négatif cumulé : 12.71
TraceGPS Issued  Coemes-Retiers  
    - Dénivelé positif cumulé : 67.63
    - Dénivelé négatif cumulé : 67.63
\end{verbatim}
\section{Fichier de synthèse des randonnées}

Écrire une fonction qui prend en entrée un nom de fichier JSON (contenant des informations sur diverses randonnées) et un client GraphHopper, et écrit dans un nouveau fichier JSON (dont le nom sera passé en paramètre de la fonction) une liste de dictionnaires contenant le nom de la randonnée et les informations de dénivelés positif et négatif), soit quelque chose du type :
\begin{verbatim}
    [
        {
          "name": "TraceGPS Le long de la quincampoix - Pire-sur-Seiche",
          "D+": 21.87,
          "D-": 21.87
        },
        {
          "name": "TraceGPS Issued  Messac - CIRCUIT DU PORT",
          "D+": 12.71,
          "D-": 12.71
        },
        {
          "name": "TraceGPS Issued  Coemes-Retiers ",
          "D+": 67.63,
          "D-": 67.63
        }
      ]
\end{verbatim}
\end{document}
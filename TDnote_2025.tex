\documentclass[11pt,a4paper]{article}
\usepackage[utf8]{inputenc}
\usepackage[T1]{fontenc}
\usepackage{tpl} %tdtp
\usepackage{aeguill}
\usepackage{epsfig,graphicx}
\usepackage{subfigure}
\usepackage{booktabs}
\usepackage{eurosym}
\usepackage{enumitem}
\usepackage{hyperref}
\makeatletter

\sorte{TD}
\siglemat{Python et Open data}
\formation{L2 MIASHS}
\titre{TD : Les villes où il fait bon vivre}
\begin{document}

\section{Préambule}

Ce travail se fait par groupes de 2 étudiants (éventuellement 3).  
Le TP est conçu pour être réalisé en 2 heures lors de la séance de TD.  
La date limite de rendu (dépôt) précisée sur CURSUS correspond à l'heure de fin du TD.

\section{Contexte}

L'objectif de ce TP est d'appliquer des compétences d'accès aux données (fichiers et APIs), de nettoyage et de croisement, et d'analyse basique pour répondre à une question concrète : \emph{où il fait bon aller demain} parmi les chefs-lieux de région français en conciliant météo et pollution.

Les jeux de données proviennent d'Open Data (INSEE pour les communes/régions) et d'APIs publiques (prévisions météo et pollution). Deux modalités d'accès aux données sont proposées (A : hors-ligne ou B : en ligne) : ne traitez la modalité B que si vous avez le temps.

\section{Énoncé}

% \subsection{But final}

À partir de données Open Data, évaluer les chefs-lieux de région français sur trois critères et produire un classement de la \emph{douceur de ville} pour le jour courant.

\subsection{Critères (pour chaque ville)}

\begin{enumerate}
    % \item \textbf{Nombre de vélos en libre-service disponibles} (plus il y en a, mieux c'est).
    \item \textbf{Température actuelle} (idéalement proche de 20~\textdegree C).
    \item \textbf{Niveau de pollution actuel} (concentration PM2.5 ; plus c'est faible, mieux c'est).
\end{enumerate}

\subsection{Données à utiliser}

\begin{description}
    \item[INSEE (fichiers fournis)] Deux fichiers :
    \begin{itemize}
        \item liste des communes avec leurs coordonnées GPS (code INSEE, nom, latitude, longitude) : \texttt{communes\_gps.json}
        \item liste des régions avec la colonne \texttt{CHEFLIEU} (code INSEE du chef-lieu) : \texttt{v\_region\_2025.csv}
    \end{itemize}
    Ces fichiers permettent d'obtenir la liste des chefs-lieux et leurs coordonnées.
    % \item[Vélos en libre-service] Deux options :
    % \begin{itemize}
    %     \item \textbf{Niveau A (hors-ligne)} : fichier CSV/JSON fourni indiquant le nombre de vélos disponibles par ville (par code INSEE ou nom de commune).
    %     \item \textbf{Niveau B (en ligne)} : accès à une API réelle (dataset Open Data) pour récupérer le nombre de vélos disponibles.
    % \end{itemize}
    \item[Météo et pollution] :
    \begin{itemize}
        \item \textbf{Niveau A (hors-ligne)} : fichiers JSON simulant les réponses d'API pour la météo et la pollution : \texttt{villes\_pollution.json} et \texttt{villes\_temperature.json}.
        \item \textbf{Niveau B (en ligne)} : appels réels aux API \texttt{OpenWeather} (Current Weather / Air Pollution) pour récupérer les données actuelles :
        \begin{itemize}\item Page de création d'une clé d'API : \url{https://home.openweathermap.org/api_keys} (nécessite d'avoir crée un compte, gratuit)
            \item Météo actuelle : \url{https://openweathermap.org/current}
            \item Pollution actuelle (PM2.5) : \url{https://openweathermap.org/api/air-pollution}
        \end{itemize}
    \end{itemize}
\end{description}

% \subsection{Contraintes}
% \begin{itemize}
%     \item La liste de villes sur laquelle travailler est limitée aux \textbf{chefs-lieux de région} (obtenus via le fichier régions $\rightarrow$ colonne \texttt{CHEFLIEU}).
%     \item Les scripts déposés ne doivent \textbf{pas} contenir de clés d'API en clair : lire les clés depuis un fichier \texttt{credentials.json} (voir section rendu).
% \end{itemize}

\section{Niveaux de réalisation}

\subsection{Niveau 1 -- Extraction et affichage}
\begin{itemize}
    \item Charger les fichiers INSEE fournis et la source météo/pollution (fichiers locaux ou APIs selon l'approche retenue).
    \item Pour chaque chef-lieu de région, afficher :
    \begin{itemize}
        % \item le nombre de vélos disponibles,
        \item la température du moment,
        \item le niveau de pollution courant (PM2.5).
    \end{itemize}
\end{itemize}

\subsection{Niveau 2 -- Notation (étoiles)}
\begin{itemize}
    \item Transformer chaque indicateur en une note sur 5 étoiles.
    \item Exemple de règle proposée : pour chaque ville, la température et la pollution sont converties en 1 à 5 étoiles selon des seuils fixes.

    Les intervalles pour la température sont évalués du plus strict au plus large : la première condition satisfaite donne la note.

    \begin{center}
    \begin{tabular}{@{}lc@{}}
    \toprule
    Température (°C) & Étoiles \\
    \midrule
    19 -- 21 & 5 \\
    18 -- 22 & 4 \\
    15 -- 25 & 3 \\
    10 -- 30 & 2 \\
    autres & 1 \\
    \bottomrule
    \end{tabular}
    \end{center}

    Pour la pollution (PM2.5 en µg/m³), les seuils sont non chevauchants et augmentent avec la concentration (plus la pollution est faible, plus la note est élevée) :

    \begin{center}
    \begin{tabular}{@{}lc@{}}
    \toprule
    PM2.5 (µg/m³) & Étoiles \\
    \midrule
    0 -- 1 & 5 \\
    1 -- 2 & 4 \\
    2 -- 5 & 3 \\
    5 -- 10 & 2 \\
    > 10 & 1 \\
    \bottomrule
    \end{tabular}
    \end{center}
    \item Afficher un tableau récapitulatif contenant, pour chaque ville, les deux notes (météo, pollution).
\end{itemize}

\subsection{Niveau 3 -- Classement final}
\begin{itemize}
    \item Combiner les deux critères (somme des étoiles) pour obtenir un score global de "douceur de vivre".
    \item Afficher la ville la mieux classée (et idéalement le Top~5).
\end{itemize}

% \subsection{Niveau 4 -- Bonus (optionnel)}
% \begin{itemize}
%     \item Affichage cartographique interactif (par exemple via \texttt{folium}) montrant la ville retenue et les indicateurs des autres chefs-lieux.
% \end{itemize}

% \section{Deux approches d'accès aux données}

% \begin{description}
%     \item[Approche A -- Hors-ligne] Toutes les données (météo, pollution) sont fournies sous forme de fichiers CSV/JSON mock (c'est-à-dire simulés).
%     \item[Approche B -- En ligne] Les données prévisions sont accédées via des API en ligne (OpenWeather). Les fichiers INSEE restent fournis en CSV.
% \end{description}

\section{Travail attendu et livrables}

Vous devez produire un \textbf{fichier Python} nommé \texttt{tp\_villes\_bon\_vivre.py} qui réalise les traitements demandés. 
Le code doit être clair, structuré en fonctions, et éviter les traitements redondants. Utilisez les structures de données appropriées (listes, dictionnaires).
Le fichier Python devra contenir en en-tête, commentés, les noms et numéros étudiants des membres du groupe.

% \subsection{Clés API}
Ne mettez jamais vos clés d'API dans les fichiers déposés. Votre script lira les clés depuis un fichier \texttt{credentials.json} (que vous ne déposerez pas), au format suivant :

\begin{verbatim}
{
    "OpenWeather": "VOTRE_CLEF_OPENWEATHER",
}
\end{verbatim}

\end{document}

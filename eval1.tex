\documentclass[11pt,a4paper]{article}
\usepackage[utf8]{inputenc}
\usepackage[T1]{fontenc}
\usepackage{tpl} %tdtp
\usepackage{aeguill}
\usepackage{epsfig,graphicx}
\usepackage{subfigure}
\usepackage{eurosym}
\usepackage{enumitem}
\usepackage{hyperref}
\makeatletter
\vspace*{1cm}



\renewcommand\thesection{Exercice \arabic{section} : }
\renewcommand{\theenumi}{Q.\arabic{section}.\arabic{enumi}}

\sorte{TD}
\siglemat{Python et Open data}
\formation{L2 MIASHS}
\titre{Evaluation 1}
\begin{document}
\normalsize \normalfont Nom : \dotfill \hspace{4cm}
  Prénom : \dotfill

Durée : 1h30 - Documents autorisés : tous

\section*{Préambule}
\begin{itemize}
    \item Dans votre dossier PythonOpenData, créez un sous-répertoire Eval1 dans lequel vous réaliserez l'ensemble des travaux.
    \item Lancer l'éditeur Visual Studio Code, et ouvrir le dossier PythonOpenData/Eval1. 
    \item Télécharger le fichier SanFrancisco\_restaurants.json dans le dossier PythonOpenData/Eval1.
  \item Copier le fichier credentials.json du TD1 dans le dossier PythonOpenData/Eval1.
  \item Créer un fichier eval1.py dans lequel vous écrirez le code du TD. 
\end{itemize}

\section{Chargement des données}

Le fichier SanFrancisco\_restaurants.json contient des informations sur des restaurants de San Francisco. Ouvrir ce fichier dans Visual Studio Code.

Pour chaque restaurant, on dispose du nom du restaurant, de ses coordonnées GPS, ainsi que des informations
sur son propriétaire. 

Par exemple le code : 
\begin{verbatim}
{
    "nom": "Jasmine Rae Bakery",
    "gps": {
      "latitude": 37.763156,
      "longitude": -122.410351
    },
    "type": "Boulangerie",
    "proprietaire": {
      "prenom": "Ewan",
      "nom": "Henderson",
      "age": 46
    }
  },
\end{verbatim}
signifie que la boulangerie "Jasmine Rae Bakery", située aux coordonnées GPS (37.763156,-122.410351) 
est tenue par Ewan Henderson, qui a 46 ans. Dans toute la suite du sujet, on nomme "restaurant" une telle entrée du fichier.



\begin{enumerate}
\item Dans le fichier eval1.py, écrire le code permettant de charger l'ensemble des restaurants dans une liste de dictionnaires, chaque dictionnaire représentant un restaurant. 
\item Afficher le nombre de restaurants disponibles.
\end{enumerate}
\newpage
\section{Etude des restaurateurs}

\textbf{Problème posé :} les patrons de pizzerias sont-ils plus âgés que les patrons de boulangerie ? 

\begin{enumerate}
    \item Ecrire une fonction qui prend en entrée la liste des restaurants, et qui renvoie la liste des types de restaurants, sans doublons. 
    \item Ecrire une fonction qui prend en entrée la liste des restaurants, une chaine de caractère donnant le type du restaurant, et qui renvoie l'âge moyen des patrons de ce type de restaurant. 
    \item En utilisant les fonctions précédentes, afficher dans le programme principal l'âge moyen des patrons selon le type de restaurant. 
    
    L'affichage attendu est : 
    \begin{verbatim}
    Age moyen des patrons de Boulangerie : 39.96 ans
    Age moyen des patrons de Pizzeria : 39.62 ans
    Age moyen des patrons de Cuisine japonaise : 43.19 ans
    \end{verbatim}
\end{enumerate}

\section{Trajet vers les restaurants}

\textbf{Problème posé :} j'habite à San Francisco, au 1830 Harrison Street. Quel est le restaurant de cuisine japonaise le plus proche de chez moi ? 

\begin{enumerate}
    \item En utilisant éventuellement des sous-fonctions liées à GraphHopper, écrire une fonction qui prend en entrée \emph{une adresse} sous forme de chaine de caractères, \emph{une liste de restaurant}, \emph{un type} de restaurant sous forme de chaine de caractère, et qui renvoie le nom restaurant du type demandé le plus proche en distance de l'adresse. 
    \item Utiliser cette fonction pour trouver le restaurant de Cuisine japonaise le plus proche du \emph{1830 Harrison Street, San Francisco}. (On s'attend à la réponse "WE BE SUSHI").
\end{enumerate}


\section{Ecriture d'un fichier}
\textbf{Problème posé :} on souhaite fabriquer un annuaire des propriétaires de restaurants, dans un fichier CSV, de la forme : 
\begin{verbatim}
firstname;name;age;shop
Ewan;Henderson;46;Jasmine Rae Bakery
Jack;White;36;McGarden Bakery
Nicholas;Matthews;51;Luigi's Pizzeria
Dario;Nicholson;39;I LUV TERIYAKI AND SUSHI
Harrison;Adams;42;LA MEJOR BAKERY
...
\end{verbatim}
Le colonnes correspondent au prénom, nom et âge de chaque propriétaire, puis du nom de leur restaurant. 

\begin{enumerate}
\item Écrire une fonction qui prend en entrée une liste de restaurants, et un nom de fichier csv, et qui écrit l'annuaire des propriétaires au format ci-dessus. 
\item Tester en créant un fichier \verb+restaurateurs.csv+.

\end{enumerate}

\section*{Remise des travaux}
Déposer sur Cursus votre fichier eval1.py.
\end{document}

\documentclass[11pt,a4paper]{article}
\usepackage[utf8]{inputenc}
\usepackage[T1]{fontenc}
\usepackage{tpl} %tdtp
\usepackage{aeguill}
\usepackage{epsfig,graphicx}
\usepackage{subfigure}
\usepackage{eurosym}
\usepackage{enumitem}
\usepackage{hyperref}
\makeatletter


% \renewcommand\thesection{Exercice \arabic{section} : }
% \renewcommand\thesubsection{Question \arabic{section}.\arabic{subsection} : }
% \renewcommand{\theenumi}{Q.\arabic{section}.\arabic{enumi}}

\sorte{TD}
\siglemat{Python et Open data}
\formation{L2 MIASHS}
\titre{TD 10-11 : Projet Cluedo}
\begin{document}


\section{Préambule}

Pour ce projet, vous devrez travailler par groupes de 2 ou 3.

Le projet sera réalisé essentiellement lors des deux dernières séances de TD, et votre présence lors de ces séances de TD est obligatoire.

Votre rendu se fera sous la forme d’un
fichier Python. Ce fichier devra être nommé td10-11.py et contenir les noms et numéros étudiant de tous
les membres du groupe commentés, en en-tête du fichier, comme dans l’exemple suivant :
\begin{verbatim}
# 22000002 Paul Machin
# 22000227 Yolène Truc

import ...
\end{verbatim}

La date limite de rendu est indiquée sur CURSUS dans l’espace de dépôt. Les fichiers déposés sur
CURSUS ne devront surtout pas contenir vos clés d’API. Celles-ci devront être lues par votre
programme dans un fichier credentials.json (que vous ne fournirez pas pour ne pas divulguer
votre clé d’API) au format :
\begin{verbatim}
{"twitter": {
    "CONSUMER_KEY":"...",
    "CONSUMER_SECRET":"...",
    "ACCESS_TOKEN":"...",
    "ACCESS_TOKEN_SECRET":"..."
    },
"graphhopper": "..."
}
\end{verbatim}



\end{document}
